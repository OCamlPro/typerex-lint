\subsection{Conventions}
For each plugins, 2 options are generated automatically:
\begin{itemize}
\item {\tt \--\--enable-PLUGIN} which allow to enable the plugin {\tt PLUGIN} \\
  (e.g. {\tt \--\--enable-plugin-typedtree});
\item {\tt \--\--disable-PLUGIN} which allow to disable the plugin {\tt PLUGIN} \\
  (e.g. {\tt \--\--disable-plugin-typedtree});
\end{itemize}
~\\
For each linters, 3 options are generated automatically:
\begin{itemize}
\item {\tt \--\--enable-PLUGIN-LINTER} which allow to enable the linter {\tt
    LINTER} \\
  (e.g. {\tt \--\--enable-plugin-parsetree.code-identifier-length});
\item {\tt \--\--disable-PLUGIN-LINTER} which allow to disable the linter {\tt
    LINTER} \\
 (e.g. {\tt \--\--disable-plugin-parsetree.code-identifier-length});
\item {\tt \--\--PLUGIN-LINTER-warnings} which allow to choose warnings raised
  by the linter {\tt LINTER} \\
  (e.g. {\tt \--\--plugin-text.code-length.warnings -A+3..5+8});
\end{itemize}

In addition to that, the plugin developers can add specific option to a plugin
or a linter: {\tt \--\--PLUGIN-OPTION} or {\tt \--\--PLUGIN-LINTER-OPTION} \\ 
(e.g. {\tt \--\--plugin-parsetree.code-identifier-length.max-identifier-length 2}).

\subsection{Kernel Arguments}
\begin{lstlisting}[language=bash,basicstyle=\tt\small,showspaces=falseshowstringspaces=false]

  --init                 Init a project
  --path DIR             Give a project dir path
  --output-txt FILE      Output results in a text file.
  --list                 List of every plugins and warnings.
  --warn-error           Every warning returns an error status code.
  --load-plugins PLUGINS Load dynamically plugins with their 
                         corresponding 'cmxs' files.
  --save-config          Save ocp-lint default config file.
  --no-db-cache          Ignore the DB feature.
  --print-only-new       Print only new warnings.
\end{lstlisting}


\ocplint{} can use a database to make style-checking a project. To create an
initial database, the user should call {\tt ocp-lint \--\--init}, otherwise, no
database will be used.

The {\tt \--\--path} argument tells \ocplint{} the directory to scan for
files to check. \ocplint{} will check all the files with extensions
{\sf .ml}, {\sf .mli}, {\sf .cmt} and {\sf .cmti}. We are currently
parallelizing the process of checking each file in a different
process: each internal process will store its results in the
(temporary or persistent) database and the calling process will
display the results at the end.

The {\tt --load-plugins FILE.cmxs} argument can be used to load a
plugin dynamically (\ocplint{} comes with a few plugins already
statically linked), and the {\sf --list} argument can be used to list
plugins and their analyses.

Options changed on the command-line can be saved in the project
configuration file using the {\tt --save-config} argument.

Several output formats can be used to print warnings: plain-text, JSON,
HTML, etc. This is easily extensible, and we plan to provide more formats.

\subsection{Plugins Specific Arguments}
\subsubsection*{Plugin On File System}
This plugin contains linters on the file system. For example, we can check if an
interface file is missing, or do some checks on files name.

\begin{lstlisting}[language=bash,basicstyle=\tt\small,showspaces=false,showstringspaces=false]
  --disable-plugin-file-system  
                            A plugin with linters on file system like 
                            interface missing, etc.
  --enable-plugin-file-system  
                            A plugin with linters on file system like 
                            interface missing, etc.

  --disable-plugin-file-system.interface-missing  
                            Enable/Disable linter "Missing interface".
  --enable-plugin-file-system.interface-missing  
                            Enable/Disable linter "Missing interface".
  --plugin-file-system.interface-missing.warnings <value> 
                            Enable/Disable warnings from 
                            "Missing interface" (current: +A)

  --disable-plugin-file-system.project-files  
                            Enable/Disable linter "File Names".
  --enable-plugin-file-system.project-files  
                            Enable/Disable linter "File Names".
  --plugin-file-system.project-files.warnings <value> 
                            Enable/Disable warnings from "File Names" 
                            (current: +A)
\end{lstlisting}

\subsubsection*{Plugin On Text}
This plugins contains linters on the source.

\begin{lstlisting}[language=bash,basicstyle=\tt\small,showspaces=false,showstringspaces=false]
  --disable-plugin-text     A plugin with linters on the source.
  --enable-plugin-text      A plugin with linters on the source.

  --disable-plugin-text.code-length  
                            Enable/Disable linter "Code Length".
  --enable-plugin-text.code-length  
                            Enable/Disable linter "Code Length".
  --plugin-text.code-length.max-line-length <int> 
                            Maximum line length
  --plugin-text.code-length.warnings <value> 
                            Enable/Disable warnings from "Code Length" 
                            (current: +A)

  --disable-plugin-text.not-that-char  
                            Enable/Disable linter "Detect use of unwanted 
                            chars in files".
  --enable-plugin-text.not-that-char  
                            Enable/Disable linter "Detect use of unwanted 
                            chars in files".
  --plugin-text.not-that-char.warnings <value> 
                            Enable/Disable warnings from "Detect use of 
                            unwanted chars in files" (current: +A)

  --disable-plugin-text.ocp-indent  
                            Enable/Disable linter "Indention with ocp-indent".
  --enable-plugin-text.ocp-indent  
                            Enable/Disable linter "Indention with ocp-indent".
  --plugin-text.ocp-indent.warnings <value> 
                            Enable/Disable warnings from "Indention with
                            ocp-indent" (current: +A)

  --disable-plugin-text.useless-space-line  
                            Enable/Disable linter "Useless space 
                            character and empty line at the end of file".
  --enable-plugin-text.useless-space-line  
                            Enable/Disable linter "Useless space 
                            character and empty line at the end of file".
  --plugin-text.useless-space-line.warnings <value> 
                            Enable/Disable warnings from "Useless space
                            character and empty line at the end of file."
                            (current: +A)
\end{lstlisting}

\subsubsection*{Plugin On Parsetree}
This plugins contains linters that take the parsetree as input.

\begin{lstlisting}[language=bash,basicstyle=\tt\small,showspaces=false,showstringspaces=false]

  --disable-plugin-parsetree  
                            A plugin with linters on parsetree.
  --enable-plugin-parsetree  
                            A plugin with linters on parsetree.

  --disable-plugin-parsetree.check-constr-args  
                            Enable/Disable linter 
                            "Check Constructor Arguments".
  --enable-plugin-parsetree.check-constr-args  
                            Enable/Disable linter 
                            "Check Constructor Arguments".

  --plugin-parsetree.check-constr-args.warnings <value> 
                            Enable/Disable warnings from 
                            "Check Constructor Arguments" (current: +A)

  --disable-plugin-parsetree.code-identifier-length  
                            Enable/Disable linter 
                            "Code Identifier Length".
  --enable-plugin-parsetree.code-identifier-length  
                            Enable/Disable linter 
                            "Code Identifier Length".
  --plugin-parsetree.code-identifier-length.max-identifier-length <int> 
                            Identifiers with a longer name will 
                            trigger a warning
  --plugin-parsetree.code-identifier-length.min-identifier-length <int> 
                            Identifiers with a shorter name will 
                            trigger a warning
  --plugin-parsetree.code-identifier-length.warnings <value> 
                            Enable/Disable warnings from 
                            "Code Identifier Length" (current: +A)

  --disable-plugin-parsetree.code-list-on-singleton  
                            Enable/Disable linter "List function 
                            on singleton".
  --enable-plugin-parsetree.code-list-on-singleton  
                            Enable/Disable linter "List function 
                            on singleton".
  --plugin-parsetree.code-list-on-singleton.warnings <value> 
                            Enable/Disable warnings from "List function 
                            on singleton" (current: +A)

  --disable-plugin-parsetree.phys-comp-allocated-lit  
                            Enable/Disable linter "Physical comparison 
                            between allocated litterals.".
  --enable-plugin-parsetree.phys-comp-allocated-lit  
                            Enable/Disable linter "Physical comparison 
                            between allocated litterals.".
  --plugin-parsetree.phys-comp-allocated-lit.warnings <value> 
                            Enable/Disable warnings from "Physical 
                            comparison between allocated litterals." 
                            (current: +A)
\end{lstlisting}

\subsubsection*{Plugin On Typedtree}
This plugin contains linters that take the typedtree as input.

\begin{lstlisting}[language=bash,basicstyle=\tt\small,showspaces=false,showstringspaces=false]
  --disable-plugin-typedtree  
                            A plugin with linters on typed tree.
  --enable-plugin-typedtree  
                            A plugin with linters on typed tree.

  --disable-plugin-typedtree.fully-qualified-identifiers  
                            Enable/Disable linter 
                            "Fully-Qualified Identifiers".
  --enable-plugin-typedtree.fully-qualified-identifiers  
                            Enable/Disable linter 
                            "Fully-Qualified Identifiers".
  --plugin-typedtree.fully-qualified-identifiers.ignore-depth <int> 
                            Ignore qualified identifiers of that depth, 
                            not fully qualified
  --plugin-typedtree.fully-qualified-identifiers.ignore-operators <bool> 
                            Ignore symbolic operators
  --plugin-typedtree.fully-qualified-identifiers.warnings <value> 
                            Enable/Disable warnings from "Fully-Qualified
                            Identifiers" (current: +A)

  --disable-plugin-typedtree.polymorphic-function  
                            Enable/Disable linter "Polymorphic function".
  --enable-plugin-typedtree.polymorphic-function  
                            Enable/Disable linter "Polymorphic function".
  --plugin-typedtree.polymorphic-function.warnings <value> 
                            Enable/Disable warnings from 
                            "Polymorphic function" (current: +A)
\end{lstlisting}

\subsubsection*{Plugin Complex}
This plugin contains linters which takes different inputs to do their checks
(e.g. parsetree and typedtree, lexer tokens and source, etc.)

\begin{lstlisting}[language=bash,basicstyle=\tt\small,showspaces=false,showstringspaces=false]

  --disable-plugin-complex  A plugin with linters on different inputs.
  --enable-plugin-complex   A plugin with linters on different inputs.

  --disable-plugin-complex.interface-module-type-name  
                            Enable/Disable linter "Checks on 
                            module type name".
  --enable-plugin-complex.interface-module-type-name  
                            Enable/Disable linter "Checks on 
                            module type name".

  --plugin-complex.interface-module-type-name.warnings <value> 
                            Enable/Disable warnings from "Checks on 
                            module type name." (current: +A)
\end{lstlisting}


\subsubsection*{Plugin On Semantic Patches}
\begin{lstlisting}[language=bash,basicstyle=\tt\small,showspaces=false,showstringspaces=false]
  --disable-plugin-patch    Detect pattern with semantic patch.
  --enable-plugin-patch     Detect pattern with semantic patch.

  --disable-plugin-patch.sempatch-lint-default  
                            Enable/Disable linter "Lint from semantic 
                            patches (default)".
  --enable-plugin-patch.sempatch-lint-default  
                            Enable/Disable linter "Lint from semantic 
                            patches (default)".

  --plugin-patch.sempatch-lint-default.warnings <value> 
                            Enable/Disable warnings from "Lint from 
                            semantic patches (default)" (current: +A)

  --disable-plugin-patch.sempatch-lint-user-defined  
                            Enable/Disable linter "Lint from semantic 
                            patches (user defined).".
  --enable-plugin-patch.sempatch-lint-user-defined  
                            Enable/Disable linter "Lint from semantic 
                            patches (user defined).".
  --plugin-patch.sempatch-lint-user-defined.warnings <value> 
                            Enable/Disable warnings from "Lint from 
                            semantic patches (user defined)." (current: +A)
\end{lstlisting}

\subsection{A Short example}

A typical example of running \ocplint{}:
\begin{lstlisting}[language=bash,basicstyle=\tt\small]
$ ocp-lint --path tools/ocp-lint --disable-plugin-typedtree
Summary:
* 11 files were linted
* 40 warnings were emitted:
 * 2 "interface_missing" number 1
 * 2 "code_length" number 1
 * 4 "ocp_indent" number 1
File "lint_input.ml", line 1:
  "ocp_indent" number 1:
   File lint_input.ml' is not indented correctly.
File "lint_actions.ml", line 1:
  "code_length" number 1:
  This line is too long ('82'): it should be at most of size '80'.
File "main.ml", line 1:
  "interface_missing" number 1:
  Missing interface for main.ml'.
\end{lstlisting}                %$
