It is difficult to forecast all the checks that users will want to
apply on their code. We implemented a large set of checks, that can be
configured on a user basis, or per project, and semantic patches can
be used to add more patterns to detect. Nevertheless, we thought it
would be useful to allow users to define their own complex checks, and
for that, we provided a simple way to develop plugins and link them at
runtime.

We sketch here the development of a simple plugin. First, we create a
\verb-Plugin- to which analyses will be attached:
\noindent\begin{lstlisting}[language=caml]
module Plugin = Lint_plugin_api.MakePlugin (struct
  let name = "Text Plugin"
  let short_name = "plugin_text"
  let details = "A plugin..."
end)
\end{lstlisting}

Then, we attach a first analysis to this plugin:
\noindent\begin{lstlisting}[language=caml]
module Linter = Plugin.MakeLint(struct
  let name = "Detect use of.."
  let version = 1
  let short_name = "not_that_char"
  let details = "Detect ..."
end)
\end{lstlisting}

We can now define and attach the warnings that will be raised.
For brievity, we just display one definition:
\noindent\begin{lstlisting}[language=caml]
let w_non_ascii_char =
  Linter.new_warning ~id:1
  ~short_name:"non_ascii_char"
  ~msg:"Non-ASCII char $char used"
\end{lstlisting}

In the code, we usually prefer the use of symbolic warnings, so we
define them and provide a translation function:
\noindent\begin{lstlisting}[language=caml]
type warning =
  | NonAsciiChar of string
  | ..
module Warnings =
 Linter.MakeWarnings(struct
  type t = warning
  let to_warning = function
  | NonAsciiChar s ->
     w_non_ascii_char, ["char", s]
  | ..
  end)
\end{lstlisting}

We can now define our analysis. We can use several
reporting functions defined by the {\sf MakeWarnings}
functor:
\noindent\begin{lstlisting}[language=caml]
let check_line lnum line file =
  ..
  Warnings.report_file_line_col file lnum i
     (NonAsciiChar (Printf.sprintf "\\%03d" c))
  ..
let check_file file =
  FileString.iteri_lines (fun lnum line ->
      check_line lnum line file) file
\end{lstlisting}

Finally, we declare that the analysis should be
carried out on source files:
\noindent\begin{lstlisting}[language=caml]
module MainSRC = Linter.MakeInputML(struct
  let main source = check_file source
end)
\end{lstlisting}

We have predefined several kinds of inputs for analysis:
\begin{itemize}
\item All files: the input is the list of files to check
\item Source file: the input is the name of a source file to check
\item Tokens: the input is the list of tokens from the OCaml lexer
\item Parsetree: the input is the standard AST
\item Typedtree: the input is the typed AST, from the \verb-.cmt- file
\end{itemize}

Currently, we have created one plugin per input, that gathers together
all the analyses on that input. However, we expect users to define
their own plugins, mixing analyses on the different inputs, as better
suited for their checks.

Finally, we are now working on \emph{custom inputs}, i.e. an input
that can be defined by the user, and then used by multiple analyses.
A typical example of use is the {\sf ocp-lint-plugin-parsing} plugin,
that includes a full OCaml 4.03 parser, that we have modified to
generate an AST closer to the real input. Currently, this plugin is
defined as one analysis, taking a source and applying many checks on
the AST (for example, to warn for extra parenthesis). However, we
would like to define the analysis separately, while still parsing only
once the file with the new parser, defined as a user custom input.
